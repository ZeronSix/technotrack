\documentclass{article}
\usepackage{amsmath}
\usepackage{amssymb}
\usepackage[utf8]{inputenc}
\usepackage[russian]{babel}
\begin{document} 

В современной математике одно из самых перспективных направлений - это квантовая постуляция поглощающих компиляций. При этом не обойтись без вычисления производных. Так, например, гироскопический оператор графа определяется следующим образом.
Оставим доказательство вывода читателю на размышление в свободное время.
\begin{equation*}
f(x) = {{x}^{2}} + {{{5} * {x}} * {\ln\left({\sin\left({x}\right)}\right)}}
\end{equation*}
Очевидно, что:
\begin{equation*}
f'(x) = {{{2} * {{x}^{1}}} * {1}} + {{{\left({{0} * {x}} + {{5} * {1}}\right)} * {\ln\left({\sin\left({x}\right)}\right)}} + {{{5} * {x}} * {{\frac{1}{\sin\left({x}\right)}} * {{\cos\left({x}\right)} * {1}}}}}
\end{equation*}
Оставим доказательство вывода читателю на размышление в свободное время.
\begin{equation*}
f'(x) = {{{2} * {x}} * {1}} + {{{\left({{0} * {x}} + {{5} * {1}}\right)} * {\ln\left({\sin\left({x}\right)}\right)}} + {{{5} * {x}} * {{\frac{1}{\sin\left({x}\right)}} * {{\cos\left({x}\right)} * {1}}}}}
\end{equation*}
Вдумчивый читатель легко догадается, что:
\begin{equation*}
f'(x) = {{2} * {x}} + {{{\left({{0} * {x}} + {{5} * {1}}\right)} * {\ln\left({\sin\left({x}\right)}\right)}} + {{{5} * {x}} * {{\frac{1}{\sin\left({x}\right)}} * {{\cos\left({x}\right)} * {1}}}}}
\end{equation*}
Оставим доказательство вывода читателю на размышление в свободное время.
\begin{equation*}
f'(x) = {{2} * {x}} + {{{\left({0} + {{5} * {1}}\right)} * {\ln\left({\sin\left({x}\right)}\right)}} + {{{5} * {x}} * {{\frac{1}{\sin\left({x}\right)}} * {{\cos\left({x}\right)} * {1}}}}}
\end{equation*}
Ясно, что:
\begin{equation*}
f'(x) = {{2} * {x}} + {{{\left({0} + {5}\right)} * {\ln\left({\sin\left({x}\right)}\right)}} + {{{5} * {x}} * {{\frac{1}{\sin\left({x}\right)}} * {{\cos\left({x}\right)} * {1}}}}}
\end{equation*}
Ясно, что:
\begin{equation*}
f'(x) = {{2} * {x}} + {{{5} * {\ln\left({\sin\left({x}\right)}\right)}} + {{{5} * {x}} * {{\frac{1}{\sin\left({x}\right)}} * {{\cos\left({x}\right)} * {1}}}}}
\end{equation*}
Вдумчивый читатель легко догадается, что:
\begin{equation*}
f'(x) = {{2} * {x}} + {{{5} * {\ln\left({\sin\left({x}\right)}\right)}} + {{{5} * {x}} * {{\frac{1}{\sin\left({x}\right)}} * {\cos\left({x}\right)}}}}
\end{equation*}
Такие дела. $\blacksquare$
\end{document}
